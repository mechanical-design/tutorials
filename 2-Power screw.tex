\documentclass[a4paper]{article}
\usepackage{enumitem}
\usepackage{amsmath,amsthm,amssymb}
\setlength{\parindent}{0pt} % Set paragraph indentation to zero

\begin{document}

\textbf{Tutorial 2: Power Screw}
\vspace{10pt}

\textbf{Question 1}

Figure Q1 shows a c-clamp used to clamp wooden blocks in a workshop. The clamp provides a compressive stress of 5.85 MPa to two blocks that are being glued together. The threaded screw is asingle start square thread having nominal diameter of 16 mm and advances 2 mm per turn. The coefficient of friction between the screw thread and the supporting threads in the frame is f = 0.25.
Determine:
\begin{enumerate}[label=(\roman*)]
    \item the root diameter, dr and mean diameter, dm of the screw
    \item the minimum force P necessary to tighten the clamp
    \item the power if the screw travel at a speed of 2 mm/s, and
    \item State the minimum value of coefficient of friction for the screw to be overhauled
\end{enumerate}

\textbf{Example Solution}



\newpage
\textbf{Q2}

The clamp assembly as shown in Figure Q2 consists of member AB and AC, which are pin connected at A. The clamp works by rotating a single start ACME thread $(\alpha=14.5^{\circ})$ with the size of 12.5 mm and pitch of 2.5 mm. At this instant, the compressive force, Fc on the wood between B and C is 180 N. The collar at the assembly has a mean diameter of 13.5 mm. Assume all the friction coefficient between all surface contracts is 0.3. Determine:

\begin{enumerate}[label=(\roman*)]
    \item the load acting at the screw.

    \item the torque required to tighten the screw.

    \item the maximum compressive force, Fc, if allowable normal stress at the screw is 10 MPa.
\end{enumerate}


\textbf{Example Solution}


\end{document}