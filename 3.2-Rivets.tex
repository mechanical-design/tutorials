\documentclass[a4paper, fleqn]{article}
\usepackage{enumitem}
\usepackage{amsmath, amssymb}
\usepackage{xstring}
\usepackage{graphicx} % For including figures
\graphicspath{{./figs/}} % Path to figures
% Set engineering notation
\providecommand{\sci}[1]{\protect\ensuremath{\times 10^{\StrSubstitute[0]{#1}{e}{}}}}
\setlength{\parindent}{0pt} % Set paragraph indentation to zero

\begin{document}

\underline{\textbf{Tutorial 3.2: Eccentric Loading}}
\vspace{10pt}

\textbf{Question 1}

\vspace{10pt}
\textbf{Example Solution}
\vspace{10pt}

\textbf{Step 1}: Find Centroid

\textbf{Step 2}: Find distance between bolts and centroid.

\textbf{Step 3}: Find tangential force

\textbf{Step 4}: Find resultant in each bolt.







% Find diameter
\textbf{Question 2}

A gusset is attached vertically to a beam to support 1kN load under it. Determine the required rivet diameter if the allowable shear stress is 70MPa.


% Find allowable load
\textbf{Question 3}

A cantilevered bracket is bolted to a column with three M14X2 bolts as shown in the figure below. Determine,

\begin{enumerate}[label=(\roman*)]
    \item Centroid of the bolts.
    \item The allowable load, P if the shearing stress in bolt 2 is limited to 143MPa.
\end{enumerate}


% Well-defined


\end{document}